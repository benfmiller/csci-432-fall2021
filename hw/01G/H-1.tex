\documentclass{article}
\usepackage{../fasy-hw}
\usepackage{hyperref}

%% UPDATE these variables:
\renewcommand{\hwnum}{1}
\title{Advanced Algorithms, Homework \hwnum}
\author{Devan Eastman-Pittam, Ben Miller}
\collab{}
\date{due: 13 September 2021}

\begin{document}

\maketitle

This homework assignment should be
submitted as a single PDF file {\bf both to D2L and to Gradescope.}

General homework expectations:
\begin{itemize}
	\item Homework should be typeset using LaTex.
	\item Answers should be in complete sentences and proofread.
	\item You will not plagiarize, nor will you share your written solutions
	      with classmates.
	\item List collaborators at the start of each question using the
	      \texttt{collab} command.
	\item Put your answers where the \texttt{todo} command currently is (and
	      remove the \texttt{todo}, but not the word \texttt{Answer}).
	\item If you are asked to come up with an algorithm, you are
	      expected to give an algorithm that beats the brute force (and, if possible, of
	      optimal time complexity). With your algorithm, please provide the following:
	      \begin{itemize}
		      \item \emph{What}: A prose explanation of the problem and the algorithm,
		            including a description of the input/output.
		      \item \emph{How}: Describe how the algorithm works, including giving
		            psuedocode for it.  Be sure to reference the pseudocode
		            from within the prose explanation.
		      \item \emph{How Fast}: Runtime, along with justification.  (Or, in the
		            extreme, a proof of termination).
		      \item \emph{Why}: Statement of the loop invariant for each loop, or
		            recursion invariant for each recursive function.
	      \end{itemize}
\end{itemize}



%%%%%%%%%%%%%%%%%%%%%%%%%%%%%%%%%%%%%%%%%%%%%%%%%%%%%%%%%%%%%%%%%%%%%%%%%%%%%%
\collab{Devan}
\nextprob{Groups}

Work in a group of size $\geq 2$.  Explain your strategy for working in a group.

\paragraph{Answer}

Our strategy for working together was a fairly standard one which quite a lot of groups will use, divide and conquer. Each of us chose a portion of the problems which we were to complete individually, then we would regroup to double check our answers, and finally after confirming our work was correct we would then integrate everything into one file.


%%%%%%%%%%%%%%%%%%%%%%%%%%%%%%%%%%%%%%%%%%%%%%%%%%%%%%%%%%%%%%%%%%%%%%%%%%%%%%
\collab{Ben}
\nextprob{More Basics}

\begin{enumerate}

	\item Use the definition of big-Theta notation to prove that $f(x)=4n-5$
	      is $\Theta(n+1)$.

	      \paragraph{Definition}
	      f is $\Theta(g)\ \iff\ f\ is\ O(g)\ and\ f\ is\ \Omega(g)$

	      i.e., $ a, b \in R \geq 0, n_{0} \in R$
	      $|\ \forall n \geq n_0$

	      $0 \leq a* g(n) \leq f(n) \leq b*g(n)$

	      \paragraph{Answer}
	      \begin{proof}
		      let $a = 1, b = 4, and\ n_0 = 2$

		      $\forall n \geq n_0, 1 \cdot n \leq 4 \cdot n - 5$

		      $\forall n \geq n_0, 4 \cdot n - 5 \leq 4 \cdot n$

		      $\therefore f(n)\ is\ \Theta(n)$

		      $\Theta(n+1)$ is equivalent to $\Theta(n)$

		      $\therefore f(n)\ is\ \Theta(n+1)$
	      \end{proof}

	\item Use induction to prove that, for all $n \in \N$, the complete graph on
	      $n$ vertices (denoted by the symbol~$K_n$) has $\frac{n(n-1)}{2}$~edges.

	      % \paragraph{Definition}
	      % A complete graph is a graph in which each pair of graph vertices is connected by an edge.
	      % \footnote{Definition from \href{https://mathworld.wolfram.com/CompleteGraph.html}{mathworld.wolfram}}

	      \paragraph{Answer}
          \begin{proof}
	      For every integer $k \geq 1$, the complete graph on
	      $n$ vertices (denoted by the symbol~$K_n$) has $\frac{n(n-1)}{2}$~edges.
	      For each new vertex added, an edge will be connected to each existing vertex.
	      In other words, for the $n^{th}$ vertex added $n-1$ edges are added. $\frac{n(n-1)}{2}$
	      can be rewritten as $\sum^{n-1}_{k=0} k$

	      {\bf Basis:} $f(1)$ = $\frac{1 (1-1)}{2}$
	      $=\frac{1(0)}{2}$
	      $=0$

	      A graph with 1 vertex has zero edges, so this is true.

          {\bf Inductuve Step: } let $k \in \mathbb{Z}\ \|\ k_{0} = 1$ suppose that f (k) is true,
          we will show that f (k+1) is also true.

          \begin{gather*}
           \sum^{(k+1)-1}_{j=0} j = ( \sum^{k-1}_{j=0} j ) + k\\
          = \frac{k(k-1)}{2} + k\\
          = \frac{k(k-1)}{2} + \frac{2k}{2}\\
          = \frac{k^2 -k + 2k}{2}\\
          = \frac{k^2 + k}{2}
          \end{gather*}

          The other side of the equation is

          \[
              \frac{(k+1)((k+1)-1)}{2} = \frac{k^2 +k}{2}

          \]

          $\therefore f(k+1)$ is true


      \end{proof}

	\item What is the negation of the following statement: for all integers $n
		      \in \Z$, $n$ is prime.

	      \paragraph{Answer}
	      $\exists\ an\ integer\ n \in \Z\ |\ n$ is not prime.

\end{enumerate}


%%%%%%%%%%%%%%%%%%%%%%%%%%%%%%%%%%%%%%%%%%%%%%%%%%%%%%%%%%%%%%%%%%%%%%%%%%%%%%
\collab{Devan}
\nextprob{Sorting before Searching?}
Let $A$ be an array of $n$ comparable objects.  We do not know if $A$ is sorted
or not.

\begin{enumerate}
	\item To answer the question \emph{is item $x$ in the $A$?}, should we
	      sort the array first?  Why or why not?

	      \paragraph{Answer}

To find if item x is in A we simply need to iterate through the array and do not need to sort it as long as the array is relatively small
this will produce the answer in linear time $O(n)$ which in a small array will be faster depending on how long it takes to sort the array, if the array
is very large it may be a good idea to sort as after sorting the search will go faster as the array can be sorted into sections where the value could exist and
could drop the time necessary to search the whole list

	\item Suppose we have $k$ elements that we want to find in $A$. Does this
	      change your answer? Why or why~not?

	      \paragraph{Answer}

We should definetely sort the array if we want to find more than a single number, such as finding all copies of a number in the array,
if the array is sorted first, when we search it will become $O(\log(n))$ time instead of $O(n)$ time which will save large amounts of time recursively
checking each value in the array

\end{enumerate}

%%%%%%%%%%%%%%%%%%%%%%%%%%%%%%%%%%%%%%%%%%%%%%%%%%%%%%%%%%%%%%%%%%%%%%%%%%%%%%
\collab{Ben}
\nextprob{Recurrence Relations}
Consider the function $T \colon \N \to \N$ defined by
\[T (n) = \begin{cases}
		1        & n=1  \\
		T(n-1)+1 & n>1.
	\end{cases}
\]
What is the closed form and asymptotic form of this recursion?  For the
closed form, use induction to prove that it is correct.  For the asymptotic
form, use the definition of big-Theta to justify.

\paragraph{Answer}

{\bf Closed Form:} $T (n) = n$

{\bf Basis:} $T (1) = 1$

{\bf Inductive Assumption:}

let $k \in \mathbb{N} \geq k_{0}\ |\ k_{0} \geq 1$

Assume that $T (k) = k$

\begin{align*}
    T (k + 1) &= T((k+1)-1) +1\\
    &= ((k+1) -1 ) +1\\
    &= k + 1
\end{align*}

$\therefore T(k+1)=k+1$

{\bf Asymptotic Form:} $T(n)=\Theta(n)$

\begin{proof}

    Let $n_{0} = 0$
$a=1$, and $b= 1$
let $n \geq n_{0}$

for all $n \geq n_{0}$, $a \cdot n \leq n \leq b \cdot n$\\
for all $n \geq n_{0}$, $1 \cdot n \leq n \leq 1 \cdot n$

$\therefore T(n)\ is\ \Theta(n)$

\end{proof}




%%%%%%%%%%%%%%%%%%%%%%%%%%%%%%%%%%%%%%%%%%%%%%%%%%%%%%%%%%%%%%%%%%%%%%%%%%%%%%
\collab{Devan}
\nextprob{More Recurrence Relations}

What is the asymptotic form of the following recurrence
relations? (Show work for partial credit, but full justification is not required
on this question).
Let $T \colon \N \to N$ be defined by $T(1)=1$ and, for $n>1$,
\begin{enumerate}
    \item $T(n) = 16 T(n/4) + n$
        \paragraph{Answer} {$a=16,b=4, \log4(16)=2,n^{\log_b(a)} = n^2,f (n)=n\\
        case 1, \epsilon=1$  \\
        $\therefore 16T(n/4) + n =\Theta(n^2)$}
    \item $T(n) = 2 T(n/2) + n \log{n}$
        \paragraph{Answer} {$a=2,b=2, \log_2(2)=1,n^{\log_b(a)}=n,f(n)=n$\\
        case 2 \\
        $\therefore 2 T(n/2) + n = \Theta(nlog(n))$}
    \item $T(n) = 6 T(n/3) + n^2 \log{n}$
        \paragraph{Answer} {$a=6,b=3,\log3(6)=\log(6)/\log(3),n^(\log(6)/\log(3)),f(n)=n^2$ \\
        case 3, $\epsilon=1/3, c=1,  n_{0}=1$ \\
        $\therefore 6 T(n/3) + n^2 = \Theta(n^2) $}
    \item $T(n) = 4 T(n/2) + n^2$
        \paragraph{Answer} {$a=4,b=2,\log2(4)=2,n ^ \logb(a) =n^2,f(n)=n^2$ \\
        case 2 \\
        $\therefore 4 T(n/2) + n^2 = \Theta(n^2\log(n))$}
    \item $T(n) = 9 T(n/3) + n$
        \paragraph{Answer} {$a=9,b=3,\log3(9)=2,n ^  \logb(a) = n^2, f(n) = n$ \\
        case 1, $\epsilon=1$ \\
        $\therefore 9 T(n/3) + n = \Theta(n^2)$}
\end{enumerate}

%%%%%%%%%%%%%%%%%%%%%%%%%%%%%%%%%%%%%%%%%%%%%%%%%%%%%%%%%%%%%%%%%%%%%%%%%%%%%%
\collab{Ben}
\nextprob{Pancakes}

Chapter 1, Problem 9, parts (a) and (b).

Suppose you are given a stack of n pancakes of different sizes. You want to
sort the pancakes so that smaller pancakes are on top of larger pancakes.
The only operation you can perform is a flip—insert a spatula under the
top k pancakes, for some integer k between 1 and n, and flip them all over.

{\bf (a) Describe an algorithm to sort an arbitrary stack of n pancakes using
O (n) flips. Exactly how many flips does your algorithm perform in the
worst case? [Hint: This problem has nothing to do with the Tower of
Hanoi.]}

\paragraph{Answer, Part (a)}

In the array representation of the pancake stack, the first index of the array
is the top pancake.

The general procedure for the algorithm will be to find the largest, unsorted pancake,
insert the spatula under it to flip it to the top, then flip it to the bottom of the stack.
Then, we will do the same for the substack not including the pancake on the bottom and repeat until it is sorted.

\begin{algorithm} \caption{\textsc{Pancake Sort} ($PancakeStack$)}\label{alg:seb}
    {\bf Input:} $PancakeStack$ is a stack of differently sized pancakes\\
    {\bf Output:} A stack of differently sized pancakes sorted by size
    \begin{algorithmic}[1]
        \State$n \gets size(PancakeStack)$
        \State$numPasses \gets 0$
        \State$biggestPancakeIndex \gets 1$
        \State$FlipOcurred \gets true$
        \While{$FlipOcurred = true$}
        \State$FlipOcurred \gets false$
        \For{$i$ in $1$ to $n - numPasses$}
        \If{$PancakeStack[i] > PancakeStack[biggestPancakeIndex]$}
        \State$biggestPancakeIndex \gets i$
        \EndIf{}
        \EndFor{}
        \If{$biggestPancakeIndex \neq n - numPasses$}
        \State$flip(biggestPancakeIndex)$
        \State$flip(n - numPasses)$
        \State$FlipOcurred \gets true$
        \EndIf{}
        \State$ numPasses \gets numPasses + 1$
        \State$biggestPancakeIndex \gets 1$
        \EndWhile{}\\
        \Return$PancakeStack$
    \end{algorithmic}
\end{algorithm}

For one pancake, 0 flips are required. For 2 pancakes, 1 flip may be required. For any other number of pancakes,
two flips may be required. One flip is required to move a pancake to the top, then another flip is required to
move it to the bottom.

If we're assuming that we humans can view the stack as a whole, visually see the largest pancake in the stack,
then perform the flips, this algorithm would run in $O (N)$ time too!

Thus, in the worst case, $2n - 3$ flips could occur.

{\bf (b) For every positive integer n, describe a stack of n pancakes that requires
Ω (n) flips to sort.}

\paragraph{Answer, Part (b)}

If one were to construct a pancake from a sorted pancake stack by alternating the pancake placement on the
top and on the bottom of the new stack, one flip would be required for every pancake except for the last one.
The new stack would be small in the middle and large on the top and bottom. This would give us $\Omega(n)$ flips.

Likewise, a new stack could be constructed in the reverse order so that it was large in the center and small on
the top and bottom, 2 flips could be required for each pancake except for the last two pancakes, which would
also give $\Omega(n)$ flips.

%%%%%%%%%%%%%%%%%%%%%%%%%%%%%%%%%%%%%%%%%%%%%%%%%%%%%%%%%%%%%%%%%%%%%%%%%%%%%%
\collab{Devan and Ben}
\nextprob{Largest Complete Subtree}

Chapter 1, Problem 37.  For this problem, please provide a full proof of
correctness (i.e., provide a proof for the recursion invariant).

\begin{algorithm} \caption{\textsc{Full Subtree} ($Tree$)}\label{alg:seb}
    {\bf Input:} Tree\\
    {\bf Output:} Largest complete Subtree
    \begin{algorithmic}[1]
		\State$global\ int=0$
		\State$global\ valid\_list=[]$
        \State$TreeLoopLeft(Tree, left\_node, right\_node)$
    \end{algorithmic}
\end{algorithm}

\begin{algorithm} \caption{\textsc{ListTraverse} ($Tree, validlist$)}\label{alg:seb}
    \begin{algorithmic}[1]
        \State$calculate(validlist)$ // checks which nodes are related by cross referencing the tree with the list, then calculates which subtree is largest
    \end{algorithmic}
\end{algorithm}


\begin{algorithm} \caption{\textsc{TreeLoopLeft} ($Tree,left\_node,right\_node$)}\label{alg:seb}
    \begin{algorithmic}[1]
        \If{$Tree\ node\ count = 0$}
            \State$Return\ null$
        \EndIf{}

        \State$LeftCheck(left\_node)$ //check if a valid left node
        \State$RightCheck(right\_node)$ //check if a valid right node
        \If{($RightCheck(right\_node)$ and $LeftCheck(left\_node)=true)$}
        \State$addpair(TreeNode, validlist)$ //add valid node to vaild list (validlist = validlist + [TreeNode])
            \State$TreeLoopLeft(Tree,left\_node,right\_node)$ //recursively
            \Else{}
            \State$TraverseRight()$ // move up tree and to the right node to check for more valid pairs or if on origin 							       //node move to right side of tree
        \EndIf{}
        \State$ListTraverse(Tree, validlist)$
    \end{algorithmic}
\end{algorithm}

{ \bf proof of correctness:}

{\bf initialization: } The algorithm finds the largest subtree given the node list. Since there is no nodes added, returns null

{\bf  maintenance:} in the middle with k nodes, list will be $(node(1), node(2) \dots node(k),node(k+1))$ since this will simply be checked by the ListTraverse function the amount of nodes doesn't matter and it will function no matter the number as the function simply checks the tree to see if the node is valid

{\bf termination: } given the algorithm will run until no nodes are left unchecked it will terminate as long as infinite nodes are not provided. Thus the algorithm is correct

\end{document}
