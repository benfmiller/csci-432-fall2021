\documentclass{article}
\usepackage{../fasy-hw}

%% UPDATE these variables:
\renewcommand{\hwnum}{1}
\title{Advanced Algorithms, Homework \hwnum}
\author{Ben Miller}
\collab{\todo{list your collaborators here}}
\date{due: 29 September 2021}

\begin{document}

\maketitle

This homework assignment should be
submitted as a single PDF file both to D2L and to Gradescope.

General homework expectations:
\begin{itemize}
    \item Homework should be typeset using LaTex.
    \item Answers should be in complete sentences and proofread.
    \item You will not plagiarize, nor will you share your written solutions
        with classmates.
    \item List collaborators at the start of each question using the
        \texttt{collab} command.
    \item Put your answers where the \texttt{todo} command currently is (and
        remove the \texttt{todo}, but not the word \texttt{Answer}).
\end{itemize}



%%%%%%%%%%%%%%%%%%%%%%%%%%%%%%%%%%%%%%%%%%%%%%%%%%%%%%%%%%%%%%%%%%%%%%%%%%%%%%
\collab{}

% --------------------------------------------------------------------------------------------

\nextprob{Chapter 2, Problem 4, (a)}

For each of the recursive definitions, give the recurrence
relation and state the recursion invariant.  For Part (a), prove that your
recursion invariant holds.  For Part (b), prove that your recursion terminates
using a decrementing function.


{\bf  (a)} Let $A[1.. m]$ and $B[1.. n]$ be two arbitrary arrays. A common subsequence
of A and B is both a subsequence of A and a subsequence of B. Give
a simple recursive definition for the function $lcs(A, B)$, which gives the
length of the longest common subsequence of A and B.

\paragraph{Answer}

If the cuurent element of A and B is equal, we up the count by one, and move on.
If they are not, we advance down B until we find an equal value and recurse. We
take the max of that value and skipping the current value in A.

\begin{algorithm} \caption{\textsc{lcs} ($A[1..n], B[1..n]$)}\label{alg:seb}
    {\bf Input:} two sequences\\
    {\bf Output:} The length of the longest common subsequence
    \begin{algorithmic}[1]
        \State$return\ LongestCommonSubsequenceRecursive(1, 1)$
    \end{algorithmic}
\end{algorithm}

\begin{algorithm} \caption{\textsc{LongestCommonSubsequenceRecursive} (i, j)}\label{alg:seb}
    {\bf Input:} indexes i and j\\
    {\bf Output:} length of longest common subsequence
    \begin{algorithmic}[1]
        \If{$i > len(A)$}
            \State$return\ 0$
        \ElsIf{$j > len(B)$}
            \State$return\ 0$
        \ElsIf{$A[i] = B[j]$}
            \State$return\ LongestCommonSubsequenceRecursive(i+1, j+1) +1$
        \Else{}
            \For{$k\ from\ j\ to\ len(B)$}
                \If{$A[i] = B[k]$}
                    \State$skip \gets LongestCommonSubsequenceRecursive(i+1, k+1) +1$
                    \State$Break$
                \EndIf{}
            \EndFor{}
            \State$take \gets LongestCommonSubsequenceRecursive(i+1, j)$
            \State$return\ max\{skip, take\}$
        \EndIf{}
    \end{algorithmic}
\end{algorithm}

{\bf Recurrence Relation: } $T(n) = 2T(n-1) + \Theta(1)$

Realistically, we would do a lot of advancing through both arrays, but oh well. This is worst case.

{\bf Recursion Invariant:}

\footnote{The original Longest Subsequence algorithm was inspired by the textbook for all of these problems}

\todo{}

% --------------------------------------------------------------------------------------------


\nextprob{Chapter 2, Problem 4, (b)}

For each of the recursive definitions, give the recurrence
relation and state the recursion invariant.

For Part (b), prove that your recursion terminates
using a decrementing function.

{\bf (b)} Let $A[1.. m]$ and $B[1.. n]$ be two arbitrary arrays. A common super-
sequence of A and B is another sequence that contains both A and B
as subsequences. Give a simple recursive definition for the function
$scs(A, B)$, which gives the length of the shortest common supersequence
of A and B.

\paragraph{Answer}

If i or j is greater than the length of the corresponding array, we return the
length of the rest of the other array. If the current position of both arrays is
equal, we take it. Else, we take the minimum of skipping either array.

\begin{algorithm} \caption{\textsc{scs} ($A[1..n], B[1..n]$)}\label{alg:seb}
    {\bf Input:} two sequences\\
    {\bf Output:} The length of the shortest supersequence
    \begin{algorithmic}[1]
        \State$return\ ShortestSuperSequenceRecursive(1, 1)$
    \end{algorithmic}
\end{algorithm}

\begin{algorithm} \caption{\textsc{ShortestSuperSequenceRecursive} (i, j)}\label{alg:seb}
    {\bf Input:} indexes i and j\\
    {\bf Output:} length of shortest supersequence
    \begin{algorithmic}[1]
        \If{$i > len(A)$}
            \State$return\ len(B) - j$
        \ElsIf{$j > len(B)$}
            \State$return\ len(A) - i$
        \ElsIf{$A[i] = B[j]$}
            \State$return\ ShortestSuperSequenceRecursive(i+1, j+1) +1$
        \EndIf{}
        \State$takeA \gets ShortestSuperSequenceRecursive(i+1, j) +1$
        \State$takeB \gets ShortestSuperSequenceRecursive(i, j+1) +1$
        \State$return\ mim\{takeA, takeB\}$
    \end{algorithmic}
\end{algorithm}

{\bf Recurrence Relation: } $T(n) = 2T(n-1) + \Theta(1)$

\todo{}

% --------------------------------------------------------------------------------------------

\nextprob{Chapter 2, Problem 4, (c)}

For each of the recursive definitions, give the recurrence
relation and state the recursion invariant.

{\bf (c)} Call a sequence $X[1.. n]$ of numbers bitonic if there is an index i with
$1 < i < n$, such that the prefix $X[1.. i]$ is increasing and the suffix
$X[i .. n]$ is decreasing. Give a simple recursive definition for the function
$lbs(A)$, which gives the length of the longest bitonic subsequence of an
arbitrary array A of integers.

\paragraph{Answer}

For this one, we recurse up to three times per recursive call. If we are increasing, we try
skiping the current value, taking it, and taking it and flipping the increasing boolean.
If we are decreasing, we do up to two calls. Skip and take.

\begin{algorithm} \caption{\textsc{lbs} ($A[1..n]$)}\label{alg:seb}
    {\bf Input:} A sequence\\
    {\bf Output:} The length of the longest bitonic subsequence
    \begin{algorithmic}[1]
        \State$A[0] \gets -\infty$
        \State$return\ LongestBitonicRecursive(0, 1, true) -1$
    \end{algorithmic}
\end{algorithm}

\begin{algorithm} \caption{\textsc{LongestBitonicRecursive} (i, j, increasing)}\label{alg:seb}
    {\bf Input:} indexes i and j, and increasing boolean\\
    {\bf Output:} length of Longest bitonic Subsequence
    \begin{algorithmic}[1]
        \If{$j > n$}
            \State$return\ 0$
        \ElsIf{$increasing = true$}
            \If{$A[i] \geq A[j]$}
                \State$return\ LongestBitonicRecursive(i, j+1, increasing)$
            \EndIf{}
        \Else{}
            \If{$A[i] \leq A[j]$}
                \State$return\ LongestBitonicRecursive(i, j+1, increasing)$
            \EndIf{}
        \EndIf{}
        \State$skip \gets LongestBitonicRecursive(i, j+1, increasing)$
        \State$take \gets LongestBitonicRecursive(j, j+1, increasing) +1$
        \If{$increasing = true$}
            \State$take \gets max\{take, LongestBitonicRecursive(j, j+1, false) +1\}$
        \EndIf{}
        \State$return\ max\{skip, take\}$
    \end{algorithmic}
\end{algorithm}

{\bf Recurrence Relation: } $T(n) = 3T(n-1) + \Theta(1)$

\todo{}

% --------------------------------------------------------------------------------------------

\nextprob{Chapter 2, Problem 4, (d)}

For each of the recursive definitions, give the recurrence
relation and state the recursion invariant.

{\bf (d)} Call a sequence $X[1.. n]$ oscillating if $X[i] < X[i + 1]$ for all even i, and
$X[i] > X[i + 1]$ for all odd i. Give a simple recursive definition for
the function $los(A)$, which gives the length of the longest oscillating
subsequence of an arbitrary array A of integers.

\paragraph{Answer}

We will calculate this like we did the first way for longest increasing subsequence, but we'll add a
flag to tell if it's increasing or decreasing (even or odd)
The initial call to the recursive algorithm gives even as false to account for the sentinel value

\begin{algorithm} \caption{\textsc{LongestOscillating} ($A[1..n]$)}\label{alg:seb}
    {\bf Input:} A sequence\\
    {\bf Output:} The length o the longest oscillating subsequence
    \begin{algorithmic}[1]
        \State$A[0] \gets - \infty$
        \State$return\ LongestOscillatingRecursive(0, 1, false) -1$
    \end{algorithmic}
\end{algorithm}

\begin{algorithm} \caption{\textsc{LongestOscillatingRecursive} (i, j, even)}\label{alg:seb}
    {\bf Input:} indexes i and j, and boolean if even\\
    {\bf Output:} length of Longest Oscillating Subsequence
    \begin{algorithmic}[1]
        \If{$j > n$}
            \State$return\ 0$
        \ElsIf{$even = true$}
            \If{$A[i] \geq A[j]$}
                \State$return\ LongestOscillatingRecursive(i, j+1, even)$
            \EndIf{}
        \Else{}
            \If{$A[i] \leq A[j]$}
                \State$return\ LongestOscillatingRecursive(i, j+1, even)$
            \EndIf{}
        \EndIf{}

        \State$skip \gets LongestOscillatingRecursive(i, j+1, even)$
        \State$take \gets LongestOscillatingRecursive(i, j+1, not even) +1$

        \State$return\ max\{skip, take\}$
    \end{algorithmic}
\end{algorithm}

{\bf Recurrence Relation: } $T(n) = 2T(n-1) + \Theta(1)$

\todo{}

% --------------------------------------------------------------------------------------------

\nextprob{Chapter 2, Problem 4, (e)}

For each of the recursive definitions, give the recurrence
relation and state the recursion invariant.

{\bf (e)} Give a simple recursive definition for the function $sos(A)$, which gives
the length of the shortest oscillating supersequence of an arbitrary array
A of integers.

\paragraph{Answer}

We create a new supersequence B, then we calculate the minimum and maximum values in A.
Then we go through each element in A and see if appending it keeps B oscillating. If it
doesn't, we append the proper min or max value first. We could just remember the last two
elements in B if we wanted to save space, too

\begin{algorithm} \caption{\textsc{sos} ($A[1..n]$)}\label{alg:seb}
    {\bf Input:} A sequence\\
    {\bf Output:} The length of the shortest oscillating supersequence
    \begin{algorithmic}[1]
        \State$B \gets NewSequence$
        \State$minVal \gets min(A)$
        \State$maxVal \gets max(A)$
        \State$return\ ShortestOscillatingSupRecursive(1) -1$
    \end{algorithmic}
\end{algorithm}

\begin{algorithm} \caption{\textsc{ShortestOscillatingSupRecursive} (i)}\label{alg:seb}
    {\bf Input:} indexes i and j, and boolean if even\\
    {\bf Output:} length of Shortest Oscillating Supersequence
    \begin{algorithmic}[1]
        \If{$i > n$}
            \State$return\ len(B)$
        \Else{}
            \If{$B[end] > B[end-1]$}
                \If{$A[i] < B[end]$}
                    \State$append\ A[i]\ to\ B$
                    \State$return\ ShortestOscillatingSupRecursive(i+1)$
                \Else{}
                    \State$append\ minVal\ to\ B$
                    \State$return\ ShortestOscillatingSupRecursive(i)$
                \EndIf{}
            \Else{}
                \If{$A[i] > B[end]$}
                    \State$append\ A[i]\ to\ B$
                    \State$return\ ShortestOscillatingSupRecursive(i+1)$
                \Else{}
                    \State$append\ maxVal\ to\ B$
                    \State$return\ ShortestOscillatingSupRecursive(i)$
                \EndIf{}
            \EndIf{}
        \EndIf{}
    \end{algorithmic}
\end{algorithm}

{\bf Recurrence Relation: } $T(n) = T(n-1) + \Theta(1)$

\todo{}

% --------------------------------------------------------------------------------------------

\nextprob{Chapter 2, Problem 4, (f)}

For each of the recursive definitions, give the recurrence
relation and state the recursion invariant.

{\bf (f)} Call a sequence $X[1.. n]$ convex if $2 \cdot X[i] < X[i-1] + X[i+1]$ for all i.
Give a simple recursive definition for the function $lxs(A)$, which gives
the length of the longest convex subsequence of an arbitrary array A of
integers

\paragraph{Answer}

For this one, we'll put $\infty$ then $-\infty$ on both sides of the array, then we'll check to see
if the next element is equal to t times the current one - the previous one. We'll just have to
remember the current element and the previous element.

\begin{algorithm} \caption{\textsc{LongestConvex} ($A[1..n]$)}\label{alg:seb}
    {\bf Input:} A sequence\\
    {\bf Output:} The length o the longest oscillating subsequence
    \begin{algorithmic}[1]
        \State$A[0] \gets \infty$
        \State$A[n+1] \gets \infty$
        \State$positiveMax = LongestOscillatingRecursive(0, 0, 1) -2$
        \State$A[0] \gets -\infty$
        \State$A[n+1] \gets -\infty$
        \State$negativeMax = LongestOscillatingRecursive(0, 0, 1) -2$
        \State$return\ max\{positiveMax, negativeMax\}$
    \end{algorithmic}
\end{algorithm}

\begin{algorithm} \caption{\textsc{LongestConvexRecursive} (i, j, k)}\label{alg:seb}
    {\bf Input:} indexes i, j, and k\\
    {\bf Output:} length of Longest Convex Subsequence
    \begin{algorithmic}[1]
        \If{$k > n$}
            \State$return\ 0$
            \ElsIf{$2 \cdot A[j] \geq A[j] + A[k]$}
            \State$return\ LongestConvexRecursive(i, j, k +1)$
        \EndIf{}
        \Else{}
            \State$skip \gets LongestConvexRecursive(i, j, k+1)$
            \State$take \gets LongestConvexRecursive(j, k, k+1) +1$
        \State$return\ max\{skip, take\}$
        \EndIf{}
    \end{algorithmic}
\end{algorithm}

{\bf Recurrence Relation: } $T(n) = 2T(n-1) + \Theta(1)$

\todo{}
\end{document}
