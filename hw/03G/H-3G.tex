\documentclass{article}
\usepackage{../fasy-hw}

%% UPDATE these variables:
\renewcommand{\hwnum}{1}
\title{Advanced Algorithms, Homework \hwnum}
\author{Ben Miller}
\collab{\todo{list your collaborators here}}
\date{due: Wednesday, 13 October 2021}

\begin{document}

\maketitle

This homework assignment should be
submitted as a single PDF file both to D2L and to Gradescope.

General homework expectations:
\begin{itemize}
    \item Homework should be typeset using LaTex.
    \item Answers should be in complete sentences and proofread.
    \item You will not plagiarize, nor will you share your written solutions
        with classmates.
    \item List collaborators at the start of each question using the
        \texttt{collab} command.
    \item Put your answers where the \texttt{todo} command currently is (and
        remove the \texttt{todo}, but not the word \texttt{Answer}).
    \item If you are asked to come up with an algorithm, you are
        expected to give an algorithm that beats the brute force (and, if possible, of
        optimal time complexity). With your algorithm, please provide the following:
        \begin{itemize}
            \item \emph{What}: A prose explanation of the problem and the algorithm,
                including a description of the input/output.
            \item \emph{How}: Describe how the algorithm works, including giving
                psuedocode for it.  Be sure to reference the pseudocode
                from within the prose explanation.
            \item \emph{How Fast}: Runtime, along with justification.  (Or, in the
                extreme, a proof of termination).
            \item \emph{Why}: Statement of the loop invariant for each loop, or
                recursion invariant for each recursive function.
        \end{itemize}
\end{itemize}

{\bf
This homework can be submitted as a group of size $n \geq 1$.
}

%%%%%%%%%%%%%%%%%%%%%%%%%%%%%%%%%%%%%%%%%%%%%%%%%%%%%%%%%%%%%%%%%%%%%%%%%%%%%%
\collab{}
\nextprob{Sugar Packet Game}

In Section 2.2 of the textbook, Dr.~Erickson describes a two-player game.
He states, ``It's not hard to prove that
as long as there are tokens on the board, at least one player can legally
move.''  Prove this statement is correct.

\paragraph{answer}
The key to this is that a player can pass if there is no legal move.

The two move rules are that each player can move their sugar packet one square if the next
square is open or they can jump over exactly one sugar packet if there is a sugar packet on
the next square.

The only situation in which a player must pass is where 2 or more opposing sugar packets are on the
squares in front of the current sugar packet. We will call one player's sugar packets "X" and the opposing
sugar packets "Y."

\begin{proof}
    If all X's are blocked (in the endzone or two packets in front of them), Player of X must skip that turn.
    Player of Y can then move on the next turn. If Player of Y cannot move, that would mean that
    either the Y's are in the endzone or they are blocked by X's.

    There is only one X for each row in X's direction of travel, likewise for Y. So, at most, only
    one row of the opposing player's direction of travel can be blocked at any particular time.
    Only one blocking piece can be blocked at a time.

    Therefore, if a player must skip a turn, the opposing player can move, which will allow the first
    player to move on the subsequent turn.
\end{proof}

%%%%%%%%%%%%%%%%%%%%%%%%%%%%%%%%%%%%%%%%%%%%%%%%%%%%%%%%%%%%%%%%%%%%%%%%%%%%%%
\collab{}
\nextprob{Repeated pattern}

Chapter 3, Problem 26.  Only one algorithm is needed.  Bonus points for a
solution that beats $O(n^5)$ though!

Describe and analyze an algorithm that finds the maximum-area rectangular
pattern that appears more than once in a given bitmap. Specifically, given
a two-dimensional array M [1 .. n, 1 .. n] of bits as input, your algorithm
should output the area of the largest repeated rectangular pattern in M .
For example, given the bitmap shown on the left in the figure below, your
algorithm should return the integer 195, which is the area of the 15 × 13
doggo. (Although it doesn’t happen in this example, the two copies of the
repeated pattern might overlap.)

(a) For full credit, describe an algorithm that runs in $O(n^{5})$ time.

(b) For extra credit, describe an algorithm that runs in $O(n^{4})$ time.

(c) For extra extra credit, describe an algorithm that runs in $O(n^{3} polylog n)$
time.

\paragraph{answer}

The method to perform this that first comes to my mind is to have four points,
the start and end of the first pattern and the start and end of the second pattern.
We will then loop through each section with the end then the start for both patterns
and compare the arrays. Then, we will shift down for both patterns, comparing the new
length against our current max.

We'll use a set to memoize our comparisons, but we could use a 4d matrix to record checks.
A set is amortized constant time for inserts and checks while the matrix would be constant time
inserts and checks but would take more memory.

\begin{algorithm} \caption{\textsc{LargestRepeatedPatter} (M[1..n,1..n])}\label{alg:seb}
    {\bf Input:} nxn grid of booleans\\
    {\bf Output:} area of largest repeated pattern
    \begin{algorithmic}[1]
        \State$largest \gets 0$
        \State$SameSet \gets emptySet$
        \State$DiffSet \gets emptySet$
        \State$Add\ all\ (i,i,i,i)\ with\ i < n\ pairs\ to\ DiffSet$
        \For{$Acolumn \gets 1\ to\ n$}
            \For{$Bcolumn \gets Acolumn\ to\ n$}
                \For{$A \gets 1\ to\ n$}
                    \For{$B\gets 1\ to\ n$}
                        \If{$BStart \neq AStart\ or\ Acolumn != Bcolumn$}
                        \State$largest \gets max(Comparer(0, A, B, 1, Acolumn, Bcolumn), largest)$
                            Comparer
                        \EndIf{}
                    \EndFor{}
                \EndFor{}
            \EndFor{}
        \EndFor{}
        \State$return\ largest$
    \end{algorithmic}
\end{algorithm}

\begin{algorithm} \caption{\textsc{Comparer} (Size, Astart, Bstart, width, Acolumn, Bcolumn)}\label{alg:seb}
    {\bf Input:} \\
    {\bf Output:}
    \begin{algorithmic}[1]
        \State$$
    \end{algorithmic}
\end{algorithm}


\todo{}

%%%%%%%%%%%%%%%%%%%%%%%%%%%%%%%%%%%%%%%%%%%%%%%%%%%%%%%%%%%%%%%%%%%%%%%%%%%%%%
\collab{}
\nextprob{Greedy Algorithm}

What is the loop invariant of the for loop for the \textsc{GreedySchedule}
algorithm in Section 4.2 of the textbook?  Please state all relevant statements
(post-condition, pre-condition, loop invariant, and loop guard).  It may be
helpful to rewrite this for loop as a while loop and work with that.  Prove that
your loop invariant is correct.

Q= the post-condition, what is true at the end (what was the loop supposed to do) \\
P= the pre-condition, what is tru going into the loop for the first time? \\
G= the loop guard, the condition that must be met to stay in the loop. *Hint: use while loops!!* \\
$L_{i}$ (or sometimes just L) = what is true upon entering the loop or attempting to enter the loop for the ith time. *must use you variables* (implicit or explicit) (think what is getting us closer to Q)

Initilization: P $\implies L_{1}$ \\
Maintenance: $G \wedge L_{i} \implies L_{i+1}$ (could break in the middle) \\
End: $\neq G \wedge L \implies Q$

\begin{algorithmic}[1]
    \State$ sort\ F\ and\ permute\ S\ to\ match$
    \State$ count \gets 1$
    \State$ X [count] \gets 1$
    \For{$i \gets 2\ to\ n$}
        \If{$S[i] > F [X [count]]$}
            \State$ count \gets count + 1$
            \State$ X [count] \gets i$
        \EndIf{}
    \EndFor{}
    \State$ return X [1 .. count]$
\end{algorithmic}

\paragraph{answer}

Q= X[1 .. count] contains the optimal/largest number of classes and their corresponding number.\\
P= S[1..n] is a list of start times and F[1..n] is a corresponding list of end times, both permuted
so they are sorted by end time. Our count is one and the first course is the first course to finish\\
G= There are courses left in the list/count is less than length of course list\\
L_{i}= X[1.. count] are the best courses so far.

{\bf Initilization:} P $\implies L_{1}$

{\bf Maintenance: } If we choose the course F[i] that finishes first, we are able to take the next course immediately afterward that finishes first. Thank goodness we don't have to worry about credits for our degree here!

We cannot take another course that is currently going on while we are in our current course, so we skip the ones that are currently going on. If we take any course other than the next course that ends, we will not be able to take any more courses than if we took the next course that ends.

Therefore, taking the next available course gives us maxCourses(F[i]) $\geq$ maxCourses(F[i+1])

{\bf End: } When $i > n$, there are no more courses left. Because we have already taken every course we could, there are no more courses to take. We process one course per loop, so we are bound to go through every course.

\end{document}
