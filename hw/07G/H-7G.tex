\documentclass{article}
\usepackage{../fasy-hw}

%% UPDATE these variables:
\renewcommand{\hwnum}{7}
\title{Advanced Algorithms, Homework \hwnum}
\author{Ben Miller}
\collab{\todo{list your collaborators here}}
\date{due: Monday, 8 December 2021}

\begin{document}

\maketitle

This homework assignment should be
submitted as a single PDF file both to D2L and to Gradescope.

General homework expectations:
\begin{itemize}
    \item Homework should be typeset using LaTex.
    \item Answers should be in complete sentences and proofread.
    \item You will not plagiarize, nor will you share your written solutions
        with classmates.
    \item List collaborators at the start of each question using the
        \texttt{collab} command.
    \item Put your answers where the \texttt{todo} command currently is (and
        remove the \texttt{todo}, but not the word \texttt{Answer}).
    \item If you are asked to come up with an algorithm, you are
        expected to give an algorithm that beats the brute force (and, if possible, of
        optimal time complexity). With your algorithm, please provide the following:
        \begin{itemize}
            \item \emph{What}: A prose explanation of the problem and the algorithm,
                including a description of the input/output.
            \item \emph{How}: Describe how the algorithm works, including giving
                psuedocode for it.  Be sure to reference the pseudocode
                from within the prose explanation.
            \item \emph{How Fast}: Runtime, along with justification.  (Or, in the
                extreme, a proof of termination).
            \item \emph{Why}: Statement of the loop invariant for each loop, or
                recursion invariant for each recursive function.
        \end{itemize}
\end{itemize}


\collab{}
\nextprob{}

The \texttt{rand()} function in the standard C library returns a
uniformly random number in \texttt{[0,RANDMAX-1]}. Does \texttt{rand}()$\mod n$
generate a number uniformly distributed in $[0,n-1]$? (Prove or disprove).

% Note I: This is the second variant in EPI 5.12.

\paragraph{Answer}
No, \texttt{rand}()$\mod n$ will not generate a number uniformly distributed in
$[0,n-1]$. This would only work in the case where RANDMAX was a multiple of n.

Let us consider the case where RANDMAX = 5 and n = 3. Let A = a set of numbers from
0 to 4. By appliying $\mod n$ to each number in A, we get [0, 1, 2, 0, 1], which
gives us the frequencies [2,2,1]. Because the rand function is uniformly destributed,
this transform will not be uniformly distributed.

If we apply the same process with RANDMAX = 6 and n = 3, we get a frequency list of
[2,2,2]. In the general case where RANDMAX is a multiple of n, the distribution of
\texttt{rand}()$\mod n$ will be uniform.

Therefore, \texttt{rand}()$\mod n$ will not always be uniformly distributed.



\collab{\todo{}}
\nextprob{}

Algorithms where we use randomization to find a deterministic answer are known
as Las Vegas algorithms.  Monte Carlo algorithms also use randomization, but
might not always give the right answer; however, they either have a high
probability of being correct or close to correct.

\begin{enumerate}[(a)]
    \item Give a Monte Carlo algorithm to estimate~$\pi$.
    \item Let $n$ be the number of random numbers used by your algorithm.
        Explain why as $n \to \infty$, the expectation of the output for your
        algorithm is~$\pi$.
    \item Implement this algorithm and plot a line graph of
        the values returned for at least $10$ values of~$n$.
\end{enumerate}

Note: Assume that there is a function \texttt{randReal}$[a,b]$ that returns a random
real number between $a$ and $b$, iid from the uniform distribution over the
interval $[a,b]$.

We create a one by one square and inscribe a circle. From all the random points
given, we map them to coordinates in the square. The ratio of points in the circle
to all points will approximate a proportion a proportion of pi.

\collab{\todo{}}
\nextprob{}

Choose an algorithm that you analyzed on a homework in this class (can be this
HW or a previous one).  Suppose you are a journalist writing about this
break-through algorithm and write a one-page summary of the algorithm for a
general audience.  Describing the problem that this algorithm solves and the
applications of the problem should be highlighted (feel free to do some
research).  Detail of the algorithm and proofs of correctness or runtime should
be only given at a very high level.

\paragraph{Answer}

Kruskal's minimum spanning tree.

Problem: Minimum spanning tree
applications: Delivery networks. Minimize total routes/roads/cables/costs. Telephone networks. Water Usage. Sewage networks
1956 discovered.

A new algorithm has entered the scene!

Joseph Kruskal

Kruskal's algorithm is no Krusty Krab!

Think about how many networks there are in the world. From roads to irrigation,
telephone wires to sewage pipes, railroads to newspapers, networks are a huge
part of the world's infrastructure. The world is vast and infrastructure is expensive.
Joseph Kruskal has just published a new "Minimum Spanning Tree" algorithm that
will drastically reduce spending everywhere.


\todo{answer here}



\collab{\todo{}}
\nextprob{Removing an Edge}

Chapter 8, Question 4, Part(a)

For any edge e in any graph G, let G minus e denote the graph obtained by
deleting e from G. Suppose we are given a graph G and two vertices s
and t. The replacement paths problem asks us to compute the shortest-path
distance from s to t in G minus e, for every edge e of G. The output is an array
of E distances, one for each edge of G.

(a) Suppose G is a directed graph, and the shortest path from vertex s to
vertex t passes through every vertex of G. Describe an algorithm to solve
this special case of the replacement paths problem in O(E log V ) time.

\paragraph{Answer}
Because the shortest path from vertex s to vertex t passes through every vertex
of G, we know that the shortest path minus e lies on some path that goes through
a subset of the vertices up to e and a subset of the vertices after e. So, if
an edge is taken out of the shortest path, we will walk back from the edge that was
taken out and we will try every edge that connects to the path after the missing
edge to see if it is shorter.

\begin{algorithm} \caption{\textsc{RemoveEdges} (G)}\label{alg:seb}
    {\bf Input:} directed graph\\
    {\bf Output:} array of E distances
    \begin{algorithmic}[1]
        \State$Path \gets Djikstra\ shortest\ path$
        \State$EArray \gets array\ with\ length\ |E|$
        \State$NotInPathSet \gets new\ set$
        \For{$edge\ e \in E$}
            \If{$e \notin Path$}
                \State$EArray[e] \gets len(Path)$
            \Else{}
                \State$NotInPathSet\ add\ e$
            \EndIf{}
        \EndFor{}
        \For{$e \in NotInPathSet$}
            \State$tempNotInPathSet \gets new\ set$
            \State$shortestPath \gets \infty$
            \State$curVertex \gets vertex\ before\ e$
            \For{$f \in edges\ from\ curVertex\ to\ vertex\ after\ e$}
                \If{$len(Path) - len(Path\ to\ )$}

                \EndIf{}



            \EndFor{}



        \EndFor{}

    \end{algorithmic}
\end{algorithm}

            TODOasdfj

\collab{\todo{}}
\nextprob{}

Chapter 10, Question 4, (Opposing Edges)
Let G be a flow network that contains an opposing pair of edges u  v and
v  u, both with positive capacity. Let G 0 be the flow network obtained from G
by decreasing the capacities of both of these edges by min{c(u  v), c(v  u)}.
In other words:

• If c(u  v) > c(v  u), change the capacity of u  v to c(u  v) − c(v  u)
and delete v  u.
344Exercises

• If c(u  v) < c(v  u), change the capacity of v  u to c(v  u) − c(u  v)
and delete u  v.

• Finally, if c(u  v) = c(v  u), delete both u  v and v  u.

(a) Prove that every maximum (s, t)-flow in G 0 is also a maximum (s, t)-flow
in G. (Thus, by simplifying every opposing pair of edges in G, we obtain
a new reduced flow network with the same maximum flow value as G.)

(b) Prove that every minimum (s, t)-cut in G is also a minimum (s, t)-cut
in G 0 and vice versa.

(c) Prove that there is at least one maximum (s, t)-flow in G that is not a
maximum (s, t)-flow in G 0

\paragraph{Answer}

Choose the maximum edge and reduce it by the minimum edge. Remove the minimum edge

\todo{answer here}

\collab{\todo{}}
\nextprob{}

Chapter 11, Question 6, (Mini-Golf)

The SPU Commuter Silence Department is installing a mini-golf course in
the basement of the See-Bull Center! The playing field is a closed polygon
bounded by m horizontal and vertical line segments, meeting at right angles.
The course has n starting points and n holes, in one-to-one correspondence.
It is always possible hit the ball along a straight line directly from each
starting point to the corresponding hole, without touching the boundary
of the playing field. (Players are not allowed to bounce golf balls off the
walls; too much glass.) The n starting points and n holes are all at distinct
locations.
Sadly, the architect’s computer crashed just as construction was about to
begin. Thanks to the herculean efforts of their sysadmins, they were able to
recover the locations of the starting points and the holes, but all information
about which starting points correspond to which holes was lost!
Describe and analyze an algorithm to compute a one-to-one correspon-
dence between the starting points and the holes that meets the straight-line
requirement, or to report that no such correspondence exists. The input
consists of the x- and y-coordinates of the m corners of the playing field, the
n starting points, and the n holes. Assume you can determine in constant
time whether two line segments intersect, given the x- and y-coordinates
of their endpoints.

\paragraph{Answer}

A backtracking algorithm truly makes the most sense to me here even though I know this
is an application of flows and cuts

\begin{algorithm} \caption{\textsc{GolfPairing} (Corners, Starts, Holes)}\label{alg:seb}
    {\bf Input:} x,y coordinates of corners, starts, and holes\\
    {\bf Output:} one to one pairing between starts and holes
    \begin{algorithmic}[1]
        \State$$
    \end{algorithmic}
\end{algorithm}

\todo{answer here}






\end{document}
